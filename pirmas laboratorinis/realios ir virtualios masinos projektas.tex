\documentclass[12pt]{article}
\usepackage[utf8]{inputenc}
\usepackage[L7x]{fontenc}
\usepackage{lmodern}
\usepackage{anyfontsize}
\usepackage{graphicx}
\begin{document}
	\begin{titlepage}
		\begin{center}
		\Huge Virtualios ir realios mašinos projektas\\
		[4cm]
		\end{center}
		\begin{flushright}
		Atiliko:
		Irmantas Varačisnkas,\\
		Karolis Šimaitis,\\
		Miglė Augustinaitė\\
		\end{flushright}
	
	\end{titlepage}
	\tableofcontents
	\clearpage
	\section{Projekto sąlygos}
	Projektuojama interaktyvi OS.\\
	\textbf{Virtualios mašinos} procesoriaus komandos operuoja su duomenimis, esančiais registruose ir ar 		atmintyje. Yra komandos duomenų persiuntimui iš atminties į registrus ir atvirkščiai, aritmetinės (sudėties, atimties, daugybos, dalybos, palyginimo), sąlyginio ir besąlyginio valdymo perdavimo, įvedimo, išvedimo, darbo su failais (atidarymo, skaitymo, rašymo, uždarymo, sunaikinimo) ir programos pabaigos komandos. Registrai yra tokie: komandų skaitiklis, bent du bendrosios paskirties registrai, požymių registras (požymius formuoja aritmetinės, o į juos reaguoja sąlyginio valdymo perdavimo komandos). Atminties dydis yra 16 blokų po 16 žodžių (žodžio ilgį pasirinkite patys).\\
	\textbf{Realios mašinos} procesorius gali dirbti dviem režimais: vartotojo ir supervizoriaus. Virtualios mašinos atmintis atvaizduojama į vartotojo atmintį naudojant puslapių transliaciją. Yra taimeris, kas tam tikrą laiko intervalą generuojantis pertraukimus. Įvedimui naudojama klaviatūra, išvedimui - ekranas. Yra išorinės atminties įrenginys - kietasis diskas.
Vartotojas, dirbantis su sistema, programas paleidžia interaktyviai, surinkdamas atitinkamą komandą. Laikoma, kad vartotojo programos yra realios mašinos kietajame diske, į kurį jos patalpinamos „išorinėmis", modelio, o ne projektuojamos OS, priemonėmis.
	\clearpage
	\section{Realios mašinos modelis}
	\subsection{Realios mašinos brėžinys}
	
	\subsection{Realios mašinos registrai}
	\begin{itemize}
	\item HLP - bet kuris aukšto lygio kalbos procesorius. Vartotojo režime HLP vykdo užduoties programą.
	\item MODE - realios mašinos rėžimo registras. Dydis - 1 baitas. Jei reikšmė 0, dirbama supervizoriaus rėžimu, jei reikšmė nėra 0, tada dirbama vartotojo rėžimu.
	\item SF - požymių registras. Dydis - 1 baitas. Parodo procesoriaus būseną po aritmetinio veiksmo.\\
	Požymių registro struktūra: X X X X X CF ZF OF
	\begin{itemize}
		\item X - nenaudojamas.
		\item CF - carry flag. Rezultatas netilpo į skaičiaus be ženklo rėžius.
		\item ZF - zero flag. Rezultatas yra nulis.
		\item OF - overflow flag. Rezultatas netilpo į skaičiaus su ženklu rėžius.
	\end{itemize}
	\item PTR - puslapių lentelės registras. Dydis - 2 baitai. Vyresnysis baitas saugo puslapių lentelės bloko numerį, jaunesnysis baitas saugo puslapių lentelės dydį.
	\item SP - steko rodyklė. Dydis - 1 baitas. Rodo į virtualios mašinos steko viršūnę.
	\item IC - instrukcijų skaitliukas. Dydis - 1 baitas. Rodo virtualios mašinos einamąją instrukciją.
	\item C - loginis trigeris. Dydis - 1 baitas. 0 yra false, visa kita yra true.
	\item R1, R2 - bendros paskirties registrai. Dydis - po 4 baitus. Skirti atlikti komandoms.
	\item Kanalų registrai. Dydis - po 1 baitą.
	\begin{itemize}
		\item CH1 - registras rodantis ar yra atliekamas persiuntimas iš išorinės atminties į realią atmintį.
		\item CH2 - registras rodantis ar yra atliekamas persiuntimas iš realios atminties į išorinę atmintį.
		\item CH3 - registras rodantis ar atliekamas įvedimas iš klaviatūros.
		\item CH4 - registras rodantis ar atliekamas išvedimas į ekraną.
	\end{itemize}
	\item SI - supervizoriaus pertraukimų registras. Dydis - 1 baitas. 
	\item PI - programinių pertraukimų registras. Dydis - 1 baitas.
	\item TI - taimerio registras. Dydis - 1 baitas.
	
	\end{itemize}	
\end{document}