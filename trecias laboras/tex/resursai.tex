\section{Resursai}
	\subsection{Resursų deskriptorius}
		\begin{enumerate}
			\item \textbf{RID} - resurso vidinis vardas.
			\item \textbf{TYPE} - resurso tipas.
			\item \textbf{PROC} - sąrašas procesų, kurie prašė šio resurso.
			\item \textbf{STATE} - resurso būsena.
			\item \textbf{PID} - procesas, sukūręs resursą.
			\item \textbf{CONTENT} - turinys.
		\end{enumerate}
	\subsection{Resursų tipai ir magic number}
		
		(INT) - resursą norima sukurti dėl programinio arba supervizorinio pertraukimo, tačiau ne dėl timer'io pertraukimo.
		(USER) - vartotojiškas procesas.
		
		Resursų tipai, jų aprašymai, susiję procesai ir šių resursų MagicNumber (nurodomas kaip numeris sąraše):\\
		
		\begin{tabular}{| p{1,5cm} |*{3} {p{2,8cm} |} p{5cm}|}
			\hline
			Magic Number	& Pavadinimas	& Laukiantysis Procesas	& Kuriantysis Procesas	& Aprašymas\\
			\hline
			1 				& OSExit		& Start\_Stop			& ProcessKiller			& OS darbo baigimas.\\
			\hline
			2				& ReadWord		& Get\_Put\_Data		& (bet koks)			& Žodis, esantis atmintyje, įrašomas į R1.\\	
			\hline
			3				& WriteWord		& Get\_Put\_Data		& (bet koks)			& Žodis, esantis R1, įrašomas į vietą, nurodytą atmintyje.\\
			\hline
			4				& NeedInput		& Chan\_2\_Device		& (INT)					& Įvedimo srautas iš vartotojo.\\
			\hline
			5				& NeedOutput	& Chan\_3\_Device		& (INT)					& Išvedimo srautas į ekraną.\\
			\hline
			6				& ReadFile		& Chan\_1\_Device		& VM, Loader			& Failo nuskaitymas.\\
			\hline
			7				& OpenFile		& Chan\_1\_Device		& VM, Loader			& Failo atidarymas.\\
			\hline
			8				& WriteFile		& Chan\_1\_Device		& VM					& Rašymas į failą.\\
			\hline
			9				& CloseFile		& Chan\_1\_Device		& VM, Loader			& Failo uždarymas.\\
			\hline
			10				& DeleteFile	& Chan\_1\_Device		& VM					& Failo ištrinimas pagal nurodytą deskriptorių.\\
			\hline
			11				& ProgramHalt	& Process\_Killer		& VM					& Programos sustabdymas.\\
			\hline
			12				& VartotojoAtmintis 	& Job\_Governor	& Start\_Stop			& Adresas, nurodantis proceso vietą atmintyje.\\
			\hline
			13				& PakrovimoPaketas		& Loader		& Start\_Stop			& Adresas, kur krauti duomenis.\\
			\hline
			14				& Interrupt		& Job\_Governor   		& Start\_Stop			& Pertraukimas.\\
			\hline		
			15				& Validation	& JCL					& Start\_Stop			& Validacija.\\
			\hline
			16				& Run			& Main\_Proc			& Start\_Stop			& Pradėti darbą.\\
			\hline 
			
		\end{tabular}
		
		
	\subsection{Resursų paskirstytojas}
		Resursų paskirstytojas suteikia paprašytus resursus procesams pagal prioritetus, jo skirstymo pabaigoje kviečiamas procesų paskirstytojas. 
	\subsection{Resursų primityvai}
		\begin{enumerate}
			\item \textbf{Kurti resursą} - procesas kuria resursą. Perduodami parametrai yra tokie: nuoroda į proceso tėvą, resurso vidinis vardas. Resursas pridedamas prie bendrojo resursų sąrašo, taip pat prie tėvo sukurtų resursų sąrašo, sukuriamas resurso elementų sąrašas ir kuriamas laukiančių procesų sąrašas.
			\item \textbf{Prašyti resurso} - procesui paprašius resurso, jis užsiblokuoja ir yra įtraukiamas į laukiančiųjų resurso procesų sąrašą.
			\item \textbf{Atlaisvinti resursą} - primityvą kviečia procesas, kuris nori nereikalingą resursą arba perduotį informaciją kitam procesui. Primityvo pabaigoje kviečiamas resursų paskirstytojas.
			\item \textbf{Naikinti resursą} - resurso deskriptorius išmetamas iš tėvo sukurtų resursų sąrašo, bendrojo resursų sąrašo, atblokuojami procesai, kurie laukė šio resurso, sunaikinamas pats deskriptorius.
		\end{enumerate}
		