\section{Resursai}
	\subsection{Resursų deskriptorius}
		\begin{enumerate}
			\item \textbf{RID} - resurso vidinis vardas.
			\item \textbf{TYPE} - resurso tipas.
			\item \textbf{PROC} - sąrašas procesų, kurie prašė šio resurso.
			\item \textbf{STATE} - resurso būsena.
			\item \textbf{PID} - procesas, sukūręs resursą.
			\item \textbf{CONTENT} - ???.
		\end{enumerate}
	\subsection{Resursų tipai ir magic number}
		
		(INT) - resursą norima sukurti dėl programinio arba supervizorinio pertraukimo, tačiau ne dėl timer'io pertraukimo.
		(USER) - vartotojiškas procesas.
		
		\textbf{MagicNumber} - tai yra skaičius, nurodantis resurso tipą.
		
		
	\subsection{Resursų paskirstytojas}
		Resursų paskirstytojas suteikia paprašytus resursus procesams pagal prioritetus, jo skirstymo pabaigoje kviečiamas procesų paskirstytojas. 
	\subsection{Resursų primityvai}
		\begin{enumerate}
			\item \textbf{Kurti resursą} - procesas kuria resursą. Perduodami parametrai yra tokie: nuoroda į proceso tėvą, resurso vidinis vardas. Resursas pridedamas prie bendrojo resursų sąrašo, taip pat prie tėvo sukurtų resursų sąrašo, sukuriamas resurso elementų sąrašas ir kuriamas laukiančių procesų sąrašas.
			\item \textbf{Prašyti resurso} - procesui paprašius resurso, jis užsiblokuoja ir yra įtraukiamas į laukiančiųjų resurso procesų sąrašą.
			\item \textbf{Atlaisvinti resursą} - primityvą kviečia procesas, kuris nori nereikalingą resursą arba perduotį informaciją kitam procesui. Primityvo pabaigoje kviečiamas resursų paskirstytojas.
			\item \textbf{Naikinti resursą} - resurso deskriptorius išmetamas iš tėvo sukurtų resursų sąrašo, bendrojo resursų sąrašo, atblokuojami procesai, kurie laukė šio resurso, sunaikinamas pats deskriptorius.
		\end{enumerate}
		